\chapter{Introduction}
\label{ch:introduction}

\textit{Approx. 10 pages}

This document forms a general structure for a thesis. Normally, background, and
solution chapter may be split into several different chapters!

\section{Introduction}

In this report we investigated what might be are the best observation noise for
different types of bandits and Goore games. By changing the bandits with adding
more arms or changing the values of the arms, then recorded the simulations to
see what difference it made to the simulation. Then check what are the best
possible observation noise are in that case.

\section{Motivation}
We wanted to better understand these types of problems and see if there was
any formula that you may use to check what kind of values the variables should have
for best performance.

\section{Goal} 
To determine the best choice for observation noise given a
specific multi-armed bandit setting or a Goore game.

\subsection{Field of research}

\section{Thesis definition/objective / Statement of the Problem}
Find the best values for the variables in Multi armed bandit and Goore game, so you
get the best results with in reasonable time. 

\section{Contributions}
Our supervisor guided us in the right directions and gave good advice on the coding.
 
\section{Target audience}
People that use multi amred bandits or goore games to solve problems. It may also be 
good read for people that do research on these types of problems, as it may give them
better understanding on how they work.

\section{Report outline / Thesis Organization}

The rest of this report is structured as follows: In chapter
\ref{ch:background} we present an overview of the background for our research.
Specifically, we define the multi-armed bandit problem, the Goore game, the
technique known as Thompson sampling, (and GP). Subsequently, in chapter
\ref{ch:solution}, we describe the methods we used. The next chapter discusses
our results, and finally we conclude with a summary and suggestions for future
work.
