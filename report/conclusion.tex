\chapter{Conclusion and future work}
\label{ch:conclusion}

\section{Conclusion}

We have discovered the values for the observation noise, \ob{}, that maximise LTS performance in a multitude of settings for the multi-armed bandit.
We have also shown how \obstar{} varies with different parameters, thereby providing an improved intuition towards how the observation noise affects performance of this strategy.

In the Goore game having more players seems to only affect the optimal observation noise when a small amount of Gaussian white noise is present, otherwise the configurations of the Goore game which we tested all required the same value for \obstar{} to perform the best.
As the performance criterion ratio gets closer to 0 (or 1), we see that \obstar{} increases.
The best observation noise value also rises as the amount of rounds $T$ is extended, since with more time each player may use more time to explore and find the best arm to pick and exploit.
With higher white noise, \obstar{} increases at what seems to be close to a linear rate.

\section{Contributions}
Our contributions consist mainly of an improved empirical and intuitive understanding of how the performance of the LTS strategy can be maximised by choosing correct observation noise.


\section{Future work}
An avenue for future work would be to use a Gaussian process for inferring the optimal observation noise for a realistic subset of possible parameter configurations for the multi-armed bandit problem and the Goore game.
Meaning that the Gaussian processes will use existing data to estimate unknown optimal observation noises.
Currently all optimal observation noise values are discovered with programs requiring a substantial running time.
Whenever a new configuration is to be used it is necessary to run the program using the new parameters.
By employing a Gaussian process it would be possible to estimate the optimal observation noise through the use of existing data.

